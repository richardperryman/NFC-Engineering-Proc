% Carleton University SCE 4th Year Project thesis style
% University of Ottawa MSc thesis style -- modifications to the report style
% modification of suthesis style of Stanford University
% Example of use:
\documentclass[12pt]{report}
\usepackage{amsmath,amssymb,amsthm}
\usepackage{SCE4YPTemplate}
\usepackage{graphicx}
\usepackage{url}

    

\begin{document}
\title{SYSC4907 Project: \\ Sensor-Based Access Control System}
\author{
    Craig Shorrocks,
    Jessica Morris,
    Richard Perryman
}
	    % Remember to use your titles
	    % Use \copyrightyear{1885} to force a particular year
	    % for the copyright statement.
\copyrightfalse % do not produce a separate copyright page
		    % otherwise use \copyrighttrue
%    \figurespagefalse % do not produce a separate figures page
%    \tablespagefalse  % do not produce a separate tables page

% Here you insert the stuff that comes before the preface
% Each preface section is contained in a \prefacesection and starts on a
% new page.  These are numbered using Roman numerals.
% If there are no such pages, do not remove the \beforepreface command
% since it creates the title page.
\beforepreface

%=================================================================================

\prefacesection{Abstract}
	This report tells you all you need to know about something.

%=================================================================================

\prefacesection{Acknowledgements}
	I would like to thank my supervisor, anyone who paid me money, gave me
	equipment, etc.

%=================================================================================

% For some reason, these seem to require that you run latex twice?

\prefaceTOC   % to print the Table of Contents
\prefaceLOF   % to print the List of Figures
\prefaceLOT   % to print the List of Tables

%=================================================================================
		            
\prefacesection{List of Abbreviations}
    
\begin{tabular}[t]{l@{\hspace*{2cm}}l}
	APDU & Application Protocol Data Unit \\
	API & Application Programming Interface \\
	NFC & Near Field Communication \\
	HMAC & Keyed-hash message authentication code \\
\end{tabular}

%=================================================================================

\endpreface
	
%=================================================================================

\chapter{Introduction}

Give an introduction to your project.  This might include:
\begin{itemize}
  \item Motivation for your project
  \item Problem you are trying to solve
  \item Scope of your project
  \item Organization of your report
\end{itemize}
You should tune this appropriately for what best suits your project.


%=================================================================================

\chapter{The Engineering Project}

%=================================================================================

\section{Health and Safety}

Using the Health and Safety Guide posted on the course webpage, students will use this section to explain how they addressed the issues of safety and health in the system that they built for their project.

%=================================================================================

\section{Engineering Professionalism}

Using their course experience of ECOR 4995 Professional Practice, students should demonstrate how their professional responsibilities were met by the goals of their project and/or during the performance of their project. 

%=================================================================================

\section{Project Management}

One of the goals of the engineering project is real experience in working on a long-term team project.  Students should explain what project management techniques or processes were used to coordinate, manage and perform their project.

%=================================================================================

\section{Individual Contributions}

This section should carefully itemize the individual contributions of each team member. Project contributions should identify which components of work were done by each individual.  Report contributions should list the author of each major section of this report.

%---------------------------------------------------------------------------------

\subsection{Project Contributions}

Give the individual contributions of the each team member towards the project.

%---------------------------------------------------------------------------------

\subsection{Report Contributions}

Give the individual contributions of the each team member towards writing the
final report.

%=================================================================================

% Consider adding Node, Android? Maybe others?
\chapter{Technical Background}

%=================================================================================

\section{NFC}

%=================================================================================

\section{Cloud Computing}

%=================================================================================

%---------------------------------------------------------------------------------

\subsection{AWS}

%=================================================================================

\section{Security}

%=================================================================================

% Jess probably should rename these!
\section{Singe-board computers}

%=================================================================================

%---------------------------------------------------------------------------------

\subsection{Raspberry pi}

%---------------------------------------------------------------------------------

\subsection{Arduino} % ?

%=================================================================================

\chapter{Business Use Cases}

%=================================================================================

\section{Online Order Secure Pickup}

%=================================================================================

\section{Central Mail Package Pickup}

%=================================================================================

\section{Long Term Storage}

%=================================================================================

\section{Service Provider}

%=================================================================================

\chapter{Problem Analysis and System Design}

%=================================================================================

\section{Overall System Analysis}

%=================================================================================

\section{NFC}

Determining how NFC communication should take place required analysis of two hardware systems: NFC card readers, as well
as mobile devices. The most desirable protocol would be able to handle the widest variety of available hardwares for
the two devices. The performance considerations between modes was fairly minimal, so preference was placed on the
portability of the solution.

%---------------------------------------------------------------------------------

\subsection{Card Readers}

% Todo: sources
Since NFC cards are primarily designed for NFC communications, there were few restrictions that stemmed from potential
choices in card reader. Since NFC communications are specified by the ISO, most cards support enough protocols that any
decision on our part would be very likely to be supported by any card that would be desirable for any other reason.

%---------------------------------------------------------------------------------

\subsection{Mobile Devices}

The two most popular operating systems for mobile devices are iOS and Android []. Since iOS devices have NFC disabled
for everything except Apple Pay[], the only option that remained was Android. Apple devices would represent a large
part of the potential market, so alternatives to NFC would have to be considered.

Among Android devices, there exist devices which have hardware support for NFC communications, and those which rely on
host-based card emulation. Devices with hardware support have a component called a Secure Element which performs all of
the communication with the external NFC terminal. Later, applications can query this element to determine the status of
the transaction, as well as other data. Devices which use host-based card emulation use a software implementation of
secure elements. Since host-based card emulation is done through software, it will run on all Android devices running
version 4.4 or greater[], which represents over 99\% of all devices currently in use.

% Find out which protocol we actually use!
Android offers an API called Beam which is the only way Anndroid devices can use NFC in active mode []. Beam,
however, does not support sending more than one message between devices. Since the information we are sending can be
fairly large in the interest of security, this was not feasible given the restrictions of the NFC protocols we used.
Further, active communications are easier to eavesdrop on, as discussed in the background section. We decided that these
costs outweighed the simplicity of the Beam API, so passive communications were chosen for the implementation.

% Discuss AID here? or in implementation?

%---------------------------------------------------------------------------------

\subsection{Protocol}

Since our NFC communications may require more data than can be fit within an Application Protocol Data Units (APDUs), we
required a protocol which would handle segmenting and recombining the message. APDUs are defined in ISO 7816-4 [] and are
the units used by ISO 14443-4 [], which describes the transmission protocol used by NFC devices. They are restricted to
256 bytes, including headers.

% Todo, some fact checking
To work around this, the hardware device connected to the shield maintains a buffer. Under ISO 14443-4, messages can be
reliably transferred, so managing this buffer is the main consideration of our protocol. The hardware determines the
maximum amount of data that can be stored in one APDU, and fills in this value into the length field of the APDU that it
sends to the Android device. Then, the Android application responds with the minimum of that much data,  and all of the
remaining data that it has to send. Once the hardware receives an amonut of data less than the potential maximum, it
deactivates the connection. In the event that the data from the Android application fits exactly into the last message
that would be sent, the protocol still works, as the application will then respond with zero data bytes.

% Todo: MSC / FSMs?

%=================================================================================

\section{Android}

%=================================================================================

\section{Hardware}

%=================================================================================

\section{Cloud}

%=================================================================================

% For the physical thing - not sure if this should be here since Michel et al did basically all of this ;)
\section{Lock Demonstration}

%=================================================================================

\chapter{System Implementation}

%=================================================================================

\section{NFC}

%=================================================================================

\section{Android}

%=================================================================================

\section{Hardware}

%=================================================================================

\section{Cloud}

%=================================================================================

\chapter{Testing and Bug Fixes}

% Not really sure what to go with here

%=================================================================================

\chapter{Conclusions}

%=================================================================================

\renewcommand{\bibname}{References}
\begin{thebibliography}{AAA}
\bibitem{ABC} T. Me and R. You, "A great result," {\em Wonderful Journal}, vol. 5, no. 9,
	      pp. 1--11, 1998.
\bibitem{XYZ} J. Him and K. Her, "An even better result that you won't believe," {\em Best Journal Ever}, vol. 4, no. 8, pp. 55--66, 2002.
\end{thebibliography}
% If you have your general bibliography in a separate file mybib
% and you wish to use the plain style (see BIBTeX)
%    \bibliographystyle{cacm}
%    \bibliography{mybib}
    \addcontentsline{toc}{chapter}{\bibname}
    

%=================================================================================

\appendix

%=================================================================================

\chapter{Extra Simulation Results}

%=================================================================================

\chapter{Review of Linear Algebra}

%=================================================================================

\end{document}
